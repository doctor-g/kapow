\begin{movedef}{Take a hit}
{
  When you \move{take a hit}, you roll+endurance lost. On a 10+, the
  Narrator chooses one.
  \begin{choices}
  \item You are hit in a vulnerable spot. Lose an extra point of Endurance.
  \item You are incapacitated (for example, unconscious, hypnotized, or panicked).
  \end{choices}
  On a 7--9, the Narrator chooses one.
  \begin{choices}
  \item You drop what you are holding.
  \item You lose your footing.
  \item You lose track of something or someone important to the scene.
  \end{choices}
  On a miss, the Narrator may choose one of the 7--9 list above,
  but this is instead of one point of the Endurance loss that
  instigated this move.
}
{
  This is an optional move that the Narrator can call for
  when a hero takes a hit outside of a rumble.
  It is primarily used in conjunction with the \move{reduce their
    Endurance} narrator move (see page~\pageref{move:reduce-endurance}).
  ``Taking a hit'' is a general term for any kind of physical
  injury, such as falling from a great height, being struck by
  a vehicle, choking on the gas from a smoke grenade,
  or any of the other sorts of physical peril that heroes tend
  to find themselves in.
}
\end{movedef}

\begin{movedef}{End of session}
{
  \move{At the end of every session}, choose one character
  whom you trust more than before. Tell that player to add +1
  to their Bond with you. If this brings them to Bond~+4, they reset
  to Bond~+1 and mark Experience. If you do not trust any of your
  teammates more than before, than chose one character in whom you
  had hoped to gain trust but did not. Tell that player to add -1 to
  their Bond with you. If this brings them to Bond~-3, they reset to
  Bond~0 and they Mark experience.
}
{
  This move is made at the end of a session, whether it's the
  conclusion of a story or a cliffhanger.
  It serves two purposes. Mechanically, it represents the changing
  relationships among heroes. Narratively, it provides an opportunity
  to recap some of the most important moments of the session.
}
\end{movedef}