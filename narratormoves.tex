% Environment arguments: Name, description

\begin{narratormove}{Separate them}
{
  Use this move to separate the members of the team or to separate
  the team from their objective.
}
\end{narratormove}

\begin{narratormove}{Capture someone}
{
  Use this move to capture someone in the scene,
  such as one of the heroes, one of their Contacts,
  or an innocent civilian.
}
\end{narratormove}

\begin{narratormove}{Announce off-screen trouble}
{
  In this move, the camera shifts momentarily away from the heroes
  to another location.
  Although this move is, by definition, something troubling, it can also
  be used to help direct the players attention in a fruitful
  direction for the story.
  Narrative tension can rise by the players gaining some insight
  into the overall shape of the villain's plot.

  \begin{example}
    The Fly walks into Mrs.\ Penguin's hotel room looking for clues
    and finds it unoccupied. She decides to investigate but rolls
    a~4. The players don't know that kidnapping Mrs.\ Penguin was
    part of the villain's plan. The Narrator decides to announce
    off-screen trouble, saying ``Meanwhile, a woman screams as she
    is pushed into a van and driven away.''
    (Notice that the kidnapping is off-screen trouble,
    but the Narrator is also directing the players to understand
    the plot.)
  \end{example}
}
\end{narratormove}

\begin{narratormove}{Foreshadow trouble}
{
  The Narrator can drop hints about the kinds of trouble
  in which the heroes may find themselves.
  If you do not have another move lined up, it's always a good
  idea to foreshadow trouble.
  
  This is also useful move for helping shape the players'
  expectations and excitement about the Villain's scheme,
  making it an excellent move to make in response to early
  failures or uncertainty.

  \begin{example}
    Black Mask and the Fly arrive downtown but are unsure what to do.
    The Narrator chooses
    to foreshadow trouble, saying, ``You hear shouting followed
    by screeching tires. What do you do?''
  \end{example}
}
\end{narratormove}

\begin{narratormove}{Take away their stuff}
{
  Many heroes carry Gear everywhere they go, and during the course of
  a game, they may accumulate additional plot items.
  This move gives you the chance to make them do without.
}
\end{narratormove}

\begin{narratormove}{Force them to make a difficult decision}
{
  Great stories are made of hard decisions.
  Will the heroes chase the villain, even if it means an innocent
  person might get hurt? Will the hero stop and explain his reckless
  driving to the police or will he let them chase him across the city?

  Keep in mind that his is a campy adventure. These difficult
  decisions should never be trolley problems: will the heroes let one
  friend or five strangers die? Instead, a good default decision
  is between the noble and the expedient. Another way to frame it
  is a choice between two goods. Will the heroes accept the cash
  reward and donate the money to charity, or will they reject the
  award and build good will in the city?
}
\end{narratormove}

\begin{narratormove}{Offer an opportunity, with or without a cost}
{  
  This allows the Narrator to give a concrete prompt to the players.
  It is not a ``difficult decision'' as in the previous move,
  but rather providing an option to act.

  \begin{example}
    The Fly is prowling through an air duct and can see the
    thugs beneath her, heading toward the exit doors.
    The players pause, uncertain, so the Narrator
    makes a move, ``You could get the drop on them if you wanted
    to rumble, but then of course they would know you are here.
    What do you do?''
  \end{example}
}
\end{narratormove}

\begin{narratormove}{Turn their move back on them}
{
  When a player fails a move, the Narrator can often turn the tables
  on them based on the situation.

  \begin{example}
    Black Mask exhorts the police officer to give him access
    to the conference room where the kidnapping took place,
    but he fails the roll. The officer suspects Black Mask
    is the culprit returning to the scene of the crime
    and calls for backup.
  \end{example}
}
\end{narratormove}

\begin{narratormove}{Reduce their Endurance}
{
  \label{move:reduce-endurance}
  Heroic adventures can strain the mind and body of the
  even the most stalwart.
  The Narrator can have players lose Endurance when the
  narrative dictates.

  This move should not be overused, but it can effectively be
  used as a timer: if players detect a constant sapping of their
  energies, they will be motivated to press forward.

  To raise the stakes, the Narrator may additionally
  call for the player to \move{take a hit} in response to
  the Endurance loss.
}
\end{narratormove}

\begin{narratormove}{Establish a cliffhanger and end the session}
{
  When playing a campaign (that is, not a one-shot session),
  use this move to set up a cliffhanger that will have
  the players excited to return to the table.
  Fight the urge that other tabletop RPGs have to end sessions
  at quiet times, such as at the tavern or a friendly castle.
  Instead, punch the tone up, emphasize over-the-top
  villainy, and end on a high note.
  Don't forget the tell the players to tune in next week.

  Once the cliffhanger is established, the players begin their
  formal debriefing by going around the table and taking the
  \move{end of session} move.
}
\end{narratormove}

\begin{narratormove}{Modify a Contact's reliability}
{
  \label{move:change-reliability}
  This move changes the reliability of a Contact, potentially
  removing it from the player's list completely.

  This move can be used to discourage players from using a Contact
  too frequently in a game, particular in response to a failed
  exhortation of the character. A Contact is not a hero after all: they
  are never a member of the team, even if they know the identities
  of the team.  

  Note that this move can also be used to add a new Contact for a hero
  by giving it a positive reliability.  If the Contact has a strong
  starting relationship with the hero, make it a +2; if it's
  relatively weak, give it +1.  Remember to establish the other
  properties of the Contact as well, such as whether they know the
  hero in their heroic identity, secret identity, or both.
}
\end{narratormove}

\begin{narratormove}{Advance a Villainous Scheme}
{
  This move activates one of the special moves related to the
  current scenario.
  Generally, the Narrator should give an off-camera description
  of what happens, depending on whether the heroes are on the
  right track or need a hint.
  
  See page~\pageref{sec:schemes} for more
  information about Villainous Schemes.
}
\end{narratormove}

