\begin{movedef}{Rumble}
{
  When you \move{rumble}, you and your opponent each lose one Endurance,
  but first roll+Mighty. On a 10+, choose two. On a 7--9, choose one.
  \begin{choices}
    \item Remove one additional Endurance from your opponent.
    \item Wrest an item from an enemy or force it to be dropped.
    \item Put up a good defense and prevent one Endurance loss.
    \item Frighten your opponent.
    \end{choices}
  Gain +1 if you \emph{use the environment}.  
  \onamiss{}
}
{
  Every good episode includes at least one tussle between the heroes
  and some stooges, and this move provides the framework for it.
  The move encapsulates a range of actions that might come up
  in a campy rumble, including breaking vases over someone's head,
  delivering a swinging kick from a chandelier,
  sword-fighting with weapons taken from a museum case,
  and of course, good old-fashioned fisticuffs.

  A Villain's lackeys share a pool of Endurance, so each point
  of Endurance damage done to them removes one of them from
  the battle.
  If a hero chooses to frighten an opponent during the rumble,
  that opponent must do their best to change their current course
  of action.
  
  The Narrator should set up a fun and iconic environment in which a rumble
  can take place. When a player declares their move, if they
  \emph{use the environment} in a way
  that no other player has yet used, they get a +1 to the roll.

  \begin{example}
    The Black Mask has the drop on some goons. The player declares, ``I will swing in on my Black Mask Rope and deliver a solid kick!''

    The Narrator responds, ``Yeah, take a +1!'' The Black Mask rolls
    a total of 11 and chooses to prevent his own Endurance loss
    and to remove an additional Endurance from the opponent.
  \end{example}


  \begin{example}
    The Black Mask is surrounded by three ruffians. He gets into the
    rumble, rolling a 7, and chooses to frighten an opponent.
    The Narrator tells him that as he slugs one of the ruffians,
    another one backs away, saying, ``This ain't worth it!'' and
    leaves the scene.
  \end{example}
}
\end{movedef}

\begin{movedef}{Heroic Feat}
{
  When you \move{attempt a heroic feat}, roll+Mighty. On a 10+, you are
  successful. On a 7--9, you are successful and choose one.
  \begin{choices}
  \item An ally or civilian is placed in immediate danger.
  \item There is an unintended side-effect.
  \item You lose one Endurance.
  \end{choices}
  \onamiss{}
}
{
  Sometimes a hero is called upon to do some heroic physical feat,
  such as throwing a rock to knock over a bottle across the room,
  holding a mechanical door open while the orphans escape the fire,
  or leaping across a gap between buildings.
  When other moves don't cut it, make a heroic feat.

  Note that the heroic feat should never replace any of the
  playbook moves. For example, you cannot use a Heroic
  Feat to slip silently out of the ropes binding your hands
  behind your back: that's the Daredevil's \move{escape bonds}
  move. 
}
\end{movedef}

\begin{movedef}{Prowl}
  {
    When you \move{prowl}, roll+Focused. On a 10+, you are undetected and take +2 forward. On a 7--9, choose one of the following.
    \begin{choices}
    \item You are undetected but are hindered; take -2 forward.
    \item You must choose between being detected or a negative consequence.
    \end{choices}
    \onamiss{}    
  }
  {
    This move allows heroes to sneak, to spy, or to lurk without being
    detected. It is often used to try to get the drop on a group
    of foes who are in an otherwise well-defended position.
    
    \begin{example}
      Black Mask decides to sneak his way into the warehouse to get
      the drop on the goons. He rolls an 8 and opts to take the choice.
      The Narrator explains, ``As you drop in through the open window,
      you see one of the goons about to pour the mind-control serum
      onto the candy bars. Do you reveal yourself to stop them
      from doing this, or do you stay in the shadows, undetected?''
    \end{example}
  }
\end{movedef}

\begin{movedef}{Race / Chase}
  {
    When you \move{race against the clock} to a target or \move{chase}
    a target, roll+Focused. On
    a 10+, you reach your target. On a 7--9, you reach your target and
    must choose one of the following.
    \begin{choices}
    \item You or an ally are placed in grave danger.
    \item You lose one Endurance.
    \end{choices}
    \onamiss{}
  }
  {
    This move comes up when the heroes are chasing down villains
    or trying to get to the bank before the bomb goes off.
    It is agnostic of the mode of transportation: from a narrative
    point of view, the pursuit is what matters, not whether it is
    in a car, in a boat, or on foot. The implication there comes in
    how the Narrator might respond to failure.

    \begin{example}
      Black Mask knocks out the last of the flunkies and turns
      around to see The Paradox fleeing the scene on foot.
      Black Mask decides to chase, and his total roll is 10:
      Black Mask grabs The Paradox and holds him until the
      police arrive to take him away.
    \end{example}

    \begin{example}
      Naturally, The Paradox has escaped from prison and kidnapped
      the mayor's daughter. Black Mask knows that The Paradox will
      sail away from the marina to his island hideout
      any moment, so he hops into the Black\-Mask\-Mo\-bile
      and races across town.
      He rolls a 5, arriving on the coast just in time to see
      The Paradox's ship moving over the horizon.
      Black Mask will have to find another way to save the
      innocent civilian.
    \end{example}
  }
\end{movedef}

\begin{movedef}{Investigate}
  {
    \label{move:investigate}
    When you \move{investigate a scene},
    roll+Intellectual. On a 10+, either \emph{ask} two of the
    following or
    \emph{deduce} a true answer to one.  On a 7--9,
    ask one of the following. The Narrator's answers will be true.    
    \begin{choices}
    \item Who was behind this?
    \item What happened here?
    \item What was the purpose of this?
    \item Who is endangered by this?
    \item What is significant here?
    \item What is my biggest threat right now?
    \item What is the best way in / out / through?
    \end{choices}
    \onamiss{}
  }
  {
    This move is used whenever a hero is trying to find clues in 
    a location or situation. The pace is up to the situation:
    an investigation may be an hour at a crime scene or it may
    be a glance around a room during a brawl.

    A player who investigates successfully has to make an important
    choice to either \emph{ask} a question about the world
    or to \emph{deduce} a truth about the world.
    When asking, the Narrator will provide answers that are true.
    Note that the Narrator does not need to extrapolate on the
    answers: asking who is behind a crime, for example, will reveal
    a name but not a motive.
    If the player chooses to deduce an answer, then they make a
    true statement about the world. Only the specific answer to the
    question is binding to the narrative, not any additional
    extrapolation.

    A player making this move may simply ask a question that is on
    their minds rather than one from the list. In this case,
    a Narrator can answer the closest question to what was asked.

    \begin{example}
      Black Mask arrives at the scene of the kidnapping and decides
      to \move{investigate}. He rolls an 11 and decides to ask
      what happened and who was behind it.
      The Narrator explains, ``You find an eyewitness who tells you
      that the culprit was wearing a garish black-and-white suit.
      He hypnotized the victim, who then entered willingly into
      a black-and-white striped van. This sounds like the work
      of The Paradox!''
    \end{example}

    \begin{example}
      The Fly is on the scene of a bank robbery and decides to
      \move{investigate}, rolling a perfect 12.
      She decides to deduce the truth of who was behind
      this, saying, ``There's only
      one Villain bold enough to rob a bank during broad daylight,
      knowing it would draw me to the scene of the crime.
      It must be my old arch-nemesis, DDT!''
    \end{example}
  }
\end{movedef}

\begin{movedef}{Scrutinize}
  {
  When you \move{scrutinize a person},  roll+Savvy. On a 10+, hold 3. On a 7--9, hold 1. While you are interacting with the target and nothing externally significant changes, spend a hold to ask one of the following.
  \begin{choices}
    \item Are they telling the truth?
    \item What are they feeling?
    \item What is their intent?
    \item What do they wish I would do?
    \end{choices}
    On a miss, hold 1 anyway and be prepared for the worst.
  }
  {
    When a hero is not sure if an NPC is being straight with them or not,
    it's a good time to scrutinize them. The hero draws upon their
    wit, charm, and worldliness to divine the truth.

    \begin{example}
      Black Mask is asking the security guard what happened
      at the museum and decides to \move{scrutinize} him.
      He rolls an 8 and marks a hold of 1. Black Mask asks,
      ``How did the criminals get into the museum?''
      The security guard responds that they must have picked the lock
      on the door, but this sounds fishy: Black Mask spends his
      hold to ask if he is telling the truth.
      The Narrator tells Black Mask that they could not have
      picked the lock: the security guard is hiding something.
    \end{example}
  }
\end{movedef}

\begin{movedef}{Exhort}
{
  When you \move{exhort an NPC to do something}, tell them what
  you want and give them a reason, then roll+Savvy. On a 10+, they
  go along with you until or unless the reason is betrayed. On a
  7--9, they will go along with you if given concrete assurance,
  collaboration, or evidence. \onamiss{}
}
{
  This move is for those times that a hero needs an NPC to do something
  they would not do otherwise. The hero needs to have some kind of
  leverage over the NPC, even if it is simply appealing to their sense
  of justice and honor (which, of course, only works on an NPC who
  has such sensibility).
 
  If a hero fails to exhort a Contact into action, the Narrator should
  seriously consider reducing the reliability of the Contact.  Even
  just doing this one can help the heroes understand that their
  Contacts are trusted allies and not tools to abuse.

  \begin{example}
    The police have already arrived at the scene of the burglary and
    have been instructed not to let anyone into the area.
    The Fly \move{exhorts} them to let her investigate the scene.
    She rolls a 9, and so she gives them assurance that she
    is working on behalf of the commissioner. This satisfies the
    police, who are aware of her stellar reputation.
  \end{example}
}
\end{movedef}

\begin{movedef}{Help or Hinder}
  {
    When you \move{help} or \move{hinder} another hero, roll+bond. On a 10+, give a +2 or -2 to their roll. On a 7--9, give a +1 or -1 to their roll. \onamiss{}
  }
  {
    A hero in a position to help an ally can do so with this move.
    Before the other hero makes a move of their own, the helping hero
    explains how they can assist, and then makes this move.
    Multiple heroes may attempt to help, but only the highest bonus
    contributes, while every failure still counts.
    
    In rare cases, a hero may wish to hinder another hero from an action.
    This works the same way as helping, except that the bonuses become
    penalties.

    \begin{example}
      Black Mask grabs his dice declares his intent to \move{rumble} with the
      villains in the department store.
      The Fly shouts, ``Hang on, I'll throw you a mannequin you can use
      to bash them around.'' This constitutes \move{help} with the Black
      Mask, so The Fly rolls, adding her Bond with her teammate,
      and gets a 10. Black Mask catches the mannequin and quips
      about making a swing at the real dummies as he gets a +2 to
      his \move{rumble} move.
    \end{example}
  }
\end{movedef}
