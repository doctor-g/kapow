\begin{principle}{It's about the heroes}
{
You are the narrator of the story, and the heroes are the protagonists.
They will face adversity, but they will always overcome in the long run.
The reason for setting up Villains and their devious Schemes is so
that the heroes have a backdrop in front of which to shine.

Keep the heroes as the center of attention, no matter what.  The other
principles may require you to give some support and structure in order
to maintain momentum (\emph{always leave breadcrumbs}, \emph{keep the
  story moving}), never take the spotlight off the heroes.
The best and most reliable way to ensure that their actions drive
the story forward is to always return to the Narrator's fundamental
prompt: ``What do you do?''
}
\end{principle}

\begin{principle}{Embrace camp}
{
The priceless emerald scarab of Ramses II is on display at the City
Museum.
The news magnate's daughter is throwing a party to attract suitors.
The local baseball team is running a charity event where there will
be hundreds of thousands of dollars in cash stuffed into a pi\~nata
for the City Children's Hospital.
Winning a surfing competition may in fact lead to world domination.

Practice your maniacal laugh and chew up the scenery.
When you have the hero right where you want them, deliver a fiendish
monologue.
Put the hero into a ridiculous deathtrap and then walk away.
Play to the tropes: the cops are good, criminals are foolish,
and a life sentence in prison just means you get to break out next
season looking for vengeance.
}
\end{principle}


\begin{principle}{Build the world together}
{
The heroes are in the spotlight, and you should have them help
you fill in the details that make the world real.
The hero creation process is designed to help get the on board
with this idea, and to help them meet the previous
principle, \textit{embrace camp}.

Another way to encourage this is to set up situations for the
heroes to talk in character. If the Narrator presents a believable
character in a scene, the other players will have the chance
to try this themselves. It also encourages players to try
to \move{scrutinize} their interlocutors, which can lead to
fun and interesting character moments that otherwise are wiped
away as the camera moves from action to action.

\begin{example}
  In their first session, Black Mask and The Fly start by receiving
  news about a high-profile kidnapping in the ritzy part of the city.
  They shout, ``Let's go check it out!'' The Narrator asks, ``How do
  you get there?'' This forces the two players to think for a moment
  about their team's particular mode of conveyance: rocket car?
  scooters? public transport?
\end{example}
}
\end{principle}


\begin{principle}{Always leave breadcrumbs}
{
A villainous scheme is essentially a mystery.
Given a whole city or world to explore, the heroes may often
be overwhelmed by options.
Always make sure that a scene contains some breadcrumbs to give them
ideas of where they can explore next.

\begin{example}
  The heroes have successfully prevented a band of hoodlums from
  kidnapping the wealthy widow from her posh hotel room, and now they
  wish to track down the Villain who was behind the plot.
  The Fly chooses to \move{investigate} the scene of the attempted
  kidnapping. Rolling an 8, she asks what is significant here.
  The Narrator informs her that in hoodlum's van they find a few
  cases of chocolate from the Bittersweet Chocolate Company.
  The heroes are not sure what this means, but they have an idea
  of where they can look next.
\end{example}

\begin{example}
  Consider the same set-up as above, but The Fly misses her roll
  with snake-eyes. The Narrator chooses to \move{announce off-screen
    trouble} and tells the heroes that the Villain has already chosen
  a back-up plan and set it into motion.
  The heroes are stymied: they are not sure what to do next and look
  to the Narrator. This allows the Narrator to take another move,
  this time, \move{foreshadow trouble}. He informs the players that
  one of the previously unconscious security guards wakes up and
  tells the heroes that he had been knocked out by being struck
  by a crate of chocolates from the Bittersweet Chocolate Company.  
\end{example}
}
\end{principle}

\begin{principle}{Keep the action moving}
{
This principle is closely related to the previous one.  Some players
will enjoy solving the riddles and puzzles that can lead them to the
Villain, but the Narrator should always be on the watch for
frustration or a lack of momentum.  This is a genre where a hero can
scratch their chin and come up with an reasonable next step, after
all, and if they can't, then a Contact or ally will step in with a
suggestion (an example of the Narrator's versatile \move{foreshadow
  trouble} move).

One impact of this principle is the suggestion to avoid the use of
miniatures or scenery. \kapow{} is not a tactical combat game. Time
and space are fluid and only serve to keep the action moving.  When
there's a battle in a museum, of course there's an antique sword ready
to swing at an enemy or an ancient vase ready to drop on their heads.
When a party splits up, one group may be in a lengthy \move{rumble}
while others \move{investigate} a scene.
}
\end{principle}


\begin{principle}{Play to find out what happens}
{
\kapow{} is fundamentally a game of collaborative storytelling,
and the heroes' experience is the only real truth.
The Narrator's notes about villainous schemes are primarily tools
for getting the story started. From there, who knows what direction
it might go? You have to play to find out.

\begin{example}
  The heroes prevented the hoodlums from kidnapping the wealthy widow,
  and they know that the hoodlums were wearing black pants and white
  turtleneck shirts.  The Fly decides to \move{investigate} and nails
  it with an 11.  She chooses to deduce a truth and shouts, ``Black
  and white? I bet there's trouble at the City Historic Theater, where
  they screen old-timey movies. Let's go!''
  The Narrator's notes indicated that the Villain was hiding out
  at the Bittersweet Chocolate Company,
  but that was just one story idea.
  There's a crescendo at the table that the Narrator wants to ride,
  so the Villain's new hideout is now the City Historic Theater,
  where there is about to be an epic showdown.
\end{example}
}
\end{principle}
